\documentclass[12pt,letterpaper]{exam}

\newcommand{\DocTitle}{Renew the DocTitle variable.}
\newcommand{\CourseNumber}{NE555}
\newcommand{\CourseName}{Nuclear Reactor Dynamics}
\newcommand{\DayTime}{TuTh 8:00-9:15}
\newcommand{\Room}{VV\&E B129}
\newcommand{\Term}{Spring 2011}
\newcommand{\Instructor}{Lewis John Lloyd}
\newcommand{\School}{University of Wisconsin - Madison}

\usepackage{listings}
\usepackage{mathtools}
\usepackage{xtab}
\usepackage{longtable}
\usepackage{array}
\usepackage{mathrsfs}
\usepackage{color}
\usepackage{multicol}
\usepackage{pdflscape}
\usepackage{pdfpages}
\usepackage{graphicx}
\usepackage{esint}
\usepackage{amsmath}
\usepackage{amssymb}
\usepackage{graphics}

\usepackage[
			pdfborder={0 0 0},
			urlcolor=cyan,
			pdftitle={\CourseNumber \DocTitle},
			pdfauthor={\Instructor}
			]{hyperref}
			
\usepackage[
			includeheadfoot,
			top=0.5in,
			bottom=0.5in,
			left=1.0in,
			right=1.0in
			]{geometry}
\setlength{\parindent}{0in}
\setlength{\parskip}{\baselineskip}

\lhead{\CourseNumber: \CourseName}
\chead{}
\rhead{\Term}
\lfoot{}
\cfoot{\thepage}
\rfoot{}
\coverlhead{\CourseNumber: \CourseName}
\coverchead{}
\coverrhead{\Term}
\coverlfoot{}
\covercfoot{\School}
\coverrfoot{}


\correctchoiceemphasis{\bfseries\color{red}}
\bracketedpoints
\pointsinmargin
%\addpoints
%\shadedsolutions

\lstset{ %
language=Matlab,                % the language of the code
basicstyle=\footnotesize,       % the size of the fonts that are used for the code
numbers=left,                   % where to put the line-numbers
numberstyle=\footnotesize,      % the size of the fonts that are used for the line-numbers
stepnumber=5,                   % the step between two line-numbers. If it's 1, each line 
                                % will be numbered
numbersep=5pt,                  % how far the line-numbers are from the code
backgroundcolor=\color{white},  % choose the background color. You must add \usepackage{color}
showspaces=false,               % show spaces adding particular underscores
showstringspaces=false,         % underline spaces within strings
showtabs=false,                 % show tabs within strings adding particular underscores
frame=single,                   % adds a frame around the code
tabsize=2,                      % sets default tabsize to 2 spaces
}

\renewcommand{\solutiontitle}{\noindent\textbf{Solution:}\par\noindent}

\newcommand{\mathsym}[1]{{}}
\newcommand{\unicode}{{}}
\newcommand{\tr}[1]{\tilde{#1}}
\newcommand{\LapTran}[1]{\mathscr{L}\left[#1\right]}
\newcommand{\InvLapTran}[1]{\mathscr{L}^{-1}\left[#1\right]}
\newcommand{\ParDer}[2]{\frac{\partial #1}{\partial #2}}
\newcommand{\Der}[2]{\frac{d\; #1}{d\;#2}}


\renewcommand{\DocTitle}{Project Handout}
\noprintanswers

%-------------------------------------------------------------------------------------
\begin{document}
%-------------------------------------------------------------------------------------
\begin{center}
\textbf{\DocTitle}
\end{center}
%-------------------------------------------------------------------------------------
\textbf{Objectives}
\begin{itemize}
\item{
To allow students to pursue their own research interest in reactor design and safety. 
}
\item{
To provide students with an opportunity to utilize the knowledge and skills obtained throughout the course in a reactor analysis calculation.
}
\item{To demonstrate to students the multi-physics nature of reactor dynamics and to familiarize them with the assumptions and limitations of the PRKE model.
}
\end{itemize}

\textbf{Tasks}

Each student team will select a reactor design subject to instructor consent.
Possible transient scenarios (startup, shutdown, accidents, etc) particular to that design will be analyzed by the team; one will be selected for simulation.
The team will then numerically simulate the transient scenario utilizing the models and methods presented in class with parameters derived either from calculations or from literature papers.

\textbf{Milestones}
\begin{center}
\setlength{\extrarowheight}{1.5pt}
\begin{longtable}{lclc}
Date&	Item &	Points \\ \hline\hline
\endhead
1/19/2011 & 1 & Group Selection 			&  		\\
1/28/2011 &	2 & Design Selection Due 		& 		\\
2/21/2011 &	3 & Project Progress Day I 		& 5 	\\
4/22/2011 & 4 & Project Progress Day II		& 5 	\\ 
5/02/2011 & 5 & Project Write-up Due 		& 10	\\
5/04/2011 & 6 & Project Presentation III	& 5		\\
5/06/2011 & 7 & Project Presentation III	& 5		\\
\end{longtable}
\end{center}
The details for each milestone are given below.
\begin{enumerate}
\item{
Students will break up into five groups of two. Each group of two will work as a team for the rest of the semester.
}
\item{
Each team is required to have submitted a reactor design to the instructor for consideration by this date.
No duplicate designs will be allowed.
This means that the sooner your group gets its idea to me, the more likely it is that you will get your first choice.
If your group is having difficulties with selecting a reactor design, please email me as soon as possible.
}
\item{
Each group will give an eight minute presentation on their particular reactor of choice.
This presentation should highlight the unique aspects of your reactor and provide the rest of the class with a basic understanding of your reactor.
Make sure to include details such as balance-of-plant features, reactivity controls, coolants, and potential accidents unique to this reactor.
}
\item{
Each group will give an eight minute presentation on various transient scenarios of interest to their reactors: sartup, shutdown, load tracking, balance-of-plant accidents, etc... 
Make sure to mention the one transient your group has chosen to model.
}
\item{
The project write-up is the finished product of your work. It should be no less than six pages (w/o graphs and figures) and no more than fifteen pages (w/ graphs and figures).
In the papers make sure to discuss the models used in your simulation and their validity.
Make sure to cite your sources.

Note: The project write-up is due prior to the final presentation.
}
\item{
The final presentations will span two days. Each group will have 15 minutes to present the their simulation results.
During the presentation, make sure to discuss which models were used and what assumptions should be kept in mind, which parameter values were used and where they were obtained, and the validity of your results.
}
\end{enumerate}

%-------------------------------------------------------------------------------------
\end{document}

