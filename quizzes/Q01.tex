\documentclass[12pt,letterpaper]{exam}

\newcommand{\DocTitle}{Renew the DocTitle variable.}
\newcommand{\CourseNumber}{NE555}
\newcommand{\CourseName}{Nuclear Reactor Dynamics}
\newcommand{\DayTime}{TuTh 8:00-9:15}
\newcommand{\Room}{VV\&E B129}
\newcommand{\Term}{Spring 2011}
\newcommand{\Instructor}{Lewis John Lloyd}
\newcommand{\School}{University of Wisconsin - Madison}

\usepackage{listings}
\usepackage{mathtools}
\usepackage{xtab}
\usepackage{longtable}
\usepackage{array}
\usepackage{mathrsfs}
\usepackage{color}
\usepackage{multicol}
\usepackage{pdflscape}
\usepackage{pdfpages}
\usepackage{graphicx}
\usepackage{esint}
\usepackage{amsmath}
\usepackage{amssymb}
\usepackage{graphics}

\usepackage[
			pdfborder={0 0 0},
			urlcolor=cyan,
			pdftitle={\CourseNumber \DocTitle},
			pdfauthor={\Instructor}
			]{hyperref}
			
\usepackage[
			includeheadfoot,
			top=0.5in,
			bottom=0.5in,
			left=1.0in,
			right=1.0in
			]{geometry}
\setlength{\parindent}{0in}
\setlength{\parskip}{\baselineskip}

\lhead{\CourseNumber: \CourseName}
\chead{}
\rhead{\Term}
\lfoot{}
\cfoot{\thepage}
\rfoot{}
\coverlhead{\CourseNumber: \CourseName}
\coverchead{}
\coverrhead{\Term}
\coverlfoot{}
\covercfoot{\School}
\coverrfoot{}


\correctchoiceemphasis{\bfseries\color{red}}
\bracketedpoints
\pointsinmargin
%\addpoints
%\shadedsolutions

\lstset{ %
language=Matlab,                % the language of the code
basicstyle=\footnotesize,       % the size of the fonts that are used for the code
numbers=left,                   % where to put the line-numbers
numberstyle=\footnotesize,      % the size of the fonts that are used for the line-numbers
stepnumber=5,                   % the step between two line-numbers. If it's 1, each line 
                                % will be numbered
numbersep=5pt,                  % how far the line-numbers are from the code
backgroundcolor=\color{white},  % choose the background color. You must add \usepackage{color}
showspaces=false,               % show spaces adding particular underscores
showstringspaces=false,         % underline spaces within strings
showtabs=false,                 % show tabs within strings adding particular underscores
frame=single,                   % adds a frame around the code
tabsize=2,                      % sets default tabsize to 2 spaces
}

\renewcommand{\solutiontitle}{\noindent\textbf{Solution:}\par\noindent}

\newcommand{\mathsym}[1]{{}}
\newcommand{\unicode}{{}}
\newcommand{\tr}[1]{\tilde{#1}}
\newcommand{\LapTran}[1]{\mathscr{L}\left[#1\right]}
\newcommand{\InvLapTran}[1]{\mathscr{L}^{-1}\left[#1\right]}
\newcommand{\ParDer}[2]{\frac{\partial #1}{\partial #2}}
\newcommand{\Der}[2]{\frac{d\; #1}{d\;#2}}


\renewcommand{\DocTitle}{Quiz 01}
\noprintanswers
\addpoints

%-------------------------------------------------------------------------------------
\begin{document}
%-------------------------------------------------------------------------------------
\begin{coverpages}
\begin{center}
\textbf{\DocTitle}
\end{center}

\makebox[\textwidth]{Name:\enspace\hrulefill}
\vfill
\flushright{
\gradetable[h]
}
\end{coverpages}
%-------------------------------------------------------------------------------------
\begin{flushleft}
\begin{center}
\fbox{\fbox{\parbox{5.5in}{\centering
Above all: Be explicit and concise.
If you run out of room for an answer, continue on the back of the page.
Partial credit will be based on how well I think you understand the physics of interest, so please be very clear.}}}
\end{center}
\vspace{0.1in}
%-------------------------------------------------------------------------------------
\begin{questions}
\question[20]{
Write down a differential equation for the time-dependent, continuous-energy, neutron diffusion equation in a multiplying media, with two source terms: an external neutron and a delayed neutron source.
Describe what each term represents.
\fullwidth{\begin{solution}
$\underbrace{\frac{1}{v}\frac{\partial \phi}{\partial t}}_{1} = 
-\underbrace{\nabla\cdot(-D\nabla\phi)}_{2} 
-\underbrace{\Sigma_t\phi}_{3} 
+ \underbrace{\int\limits_0^{\infty}\Sigma_s(E'\rightarrow E)\phi(E')dE'}_{4}
+\underbrace{\chi(E)\int\limits_0^{\infty}\nu(E')\Sigma_f(E')\phi(E')dE'}_{5}$
\begin{enumerate}
\item{$\displaystyle\frac{1}{v}\frac{\partial \phi}{\partial t}$
represents the time rate of change of the total number of neutrons.
}
\item{$\displaystyle-\nabla\cdot(-D\nabla\phi)$
represents the divergence of the neutron current within a control volume.
}
\item{$\displaystyle-\Sigma_t\phi$
represents the rate at neutrons are removed from one particular energy via all possible interactions with other particles within a control volume.
}
\item{$\displaystyle\int\limits_0^{\infty}\Sigma_s(E'\rightarrow E)\phi(E')dE'$
represents the rate at which neutrons are scattered into one particular energy from all other energies within a control volume.
}
\item{$\displaystyle\chi(E)\int\limits_0^{\infty}\nu(E')\Sigma_f(E')\phi(E')dE'$
represents the rate at which neutrons of a given energy are being produced through fission within the control volume.
}
\end{enumerate}
\end{solution}}}
\pagebreak
%-------------------------------------------------------------------------------------
\question[20]{
Starting from your solution to problem 1.
Using the following separation of variables, $\phi(\textbf{x},E,t)\, =\, P(t)\,f(\textbf{x})\,g(E)$, formally integrate the equation over energy and space. 
$f(\textbf{x})$ and $g(E)$ are normalized space and energy distribution functions.
Be \textbf{SURE} to clearly highlight any assumptions that you make along the way.
Make sure to define any variables you use.

\fullwidth{\begin{solution}

\end{solution}}}
\pagebreak
%-------------------------------------------------------------------------------------
\question[20]{
Starting from the six group point reactor kinetics equations without external sources, collapse the equations into the one delayed group point reactor kinetics equations.
Use a method that accurately captures the delayed neutron source, not the total delayed neutron precursor population.

\fullwidth{\begin{solution}[0.75in]

\end{solution}}}
\pagebreak
%-------------------------------------------------------------------------------------
\question[20]{
Solve for $y(t)$ from the following system of differential equations using Laplace Transforms.
Note: You don't need a closed form solution for $x(t)$, just $y(t)$.
\begin{minipage}{0.4\linewidth}\begin{eqnarray*}
\frac{dy}{dt} & = & 4\,y(t) + 2\,x(t) \\
\frac{dx}{dt} & = & 2\,y(t) + 4\,x(t)
\end{eqnarray*}\end{minipage}
,
\begin{minipage}{0.4\linewidth}\begin{eqnarray*}
y(0) & = & 1 \\
x(0) & = & 5
\end{eqnarray*}\end{minipage}
\fullwidth{\begin{solution}[0.75in]

\end{solution}}}
\pagebreak
%-------------------------------------------------------------------------------------
\question[10]{
What is the difference between $\lambda_{ave}$ and $\lambda_{inv}$ in both formulation and meaning?
\fullwidth{\begin{solution}[0.75in]

\end{solution}}}
%-------------------------------------------------------------------------------------
\question[10]{
What is the Inhour Equation for the 6DG PRKE?
\fullwidth{\begin{solution}[0.75in]

\end{solution}}}
%-------------------------------------------------------------------------------------
\end{questions}\end{flushleft}
\pagebreak
\thispagestyle{head}
\begin{center}
\textbf{Laplace Transforms}
\end{center}


\textbf{Function} \hfill \textbf{Transform}

$1$ \hfill $\displaystyle\frac{1}{s}$

$a$, a is a constant \hfill $\displaystyle\frac{a}{s}$

$\delta(t-\tau)$, $\delta$ is the Dirac Delta function \hfill $e^{-\tau\, s}$

$H(t-\tau)$, H is the Heaviside function \hfill $\displaystyle\frac{e^{-\tau\,s}}{s}$

$t\;H(t)$ \hfill $\displaystyle\frac{1}{s^2}$

$e^{at}$ \hfill $\displaystyle\frac{1}{s-a}$

$sin(a t)$ \hfill $\displaystyle\frac{a}{s^2+a^2}$

$cos(a t)$ \hfill $\displaystyle\frac{s}{s^2+a^2}$

$f(t)$ \hfill $\tilde{f}(s)$

$\displaystyle\frac{df(t)}{dt}$ \hfill $s \tilde{f}(s) - f(0)$









\end{document}