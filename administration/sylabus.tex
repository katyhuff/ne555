\documentclass[12pt,letterpaper]{exam}

\newcommand{\DocTitle}{Renew the DocTitle variable.}
\newcommand{\CourseNumber}{NE555}
\newcommand{\CourseName}{Nuclear Reactor Dynamics}
\newcommand{\DayTime}{TuTh 8:00-9:15}
\newcommand{\Room}{VV\&E B129}
\newcommand{\Term}{Spring 2011}
\newcommand{\Instructor}{Lewis John Lloyd}
\newcommand{\School}{University of Wisconsin - Madison}

\usepackage{listings}
\usepackage{mathtools}
\usepackage{xtab}
\usepackage{longtable}
\usepackage{array}
\usepackage{mathrsfs}
\usepackage{color}
\usepackage{multicol}
\usepackage{pdflscape}
\usepackage{pdfpages}
\usepackage{graphicx}
\usepackage{esint}
\usepackage{amsmath}
\usepackage{amssymb}
\usepackage{graphics}

\usepackage[
			pdfborder={0 0 0},
			urlcolor=cyan,
			pdftitle={\CourseNumber \DocTitle},
			pdfauthor={\Instructor}
			]{hyperref}
			
\usepackage[
			includeheadfoot,
			top=0.5in,
			bottom=0.5in,
			left=1.0in,
			right=1.0in
			]{geometry}
\setlength{\parindent}{0in}
\setlength{\parskip}{\baselineskip}

\lhead{\CourseNumber: \CourseName}
\chead{}
\rhead{\Term}
\lfoot{}
\cfoot{\thepage}
\rfoot{}
\coverlhead{\CourseNumber: \CourseName}
\coverchead{}
\coverrhead{\Term}
\coverlfoot{}
\covercfoot{\School}
\coverrfoot{}


\correctchoiceemphasis{\bfseries\color{red}}
\bracketedpoints
\pointsinmargin
%\addpoints
%\shadedsolutions

\lstset{ %
language=Matlab,                % the language of the code
basicstyle=\footnotesize,       % the size of the fonts that are used for the code
numbers=left,                   % where to put the line-numbers
numberstyle=\footnotesize,      % the size of the fonts that are used for the line-numbers
stepnumber=5,                   % the step between two line-numbers. If it's 1, each line 
                                % will be numbered
numbersep=5pt,                  % how far the line-numbers are from the code
backgroundcolor=\color{white},  % choose the background color. You must add \usepackage{color}
showspaces=false,               % show spaces adding particular underscores
showstringspaces=false,         % underline spaces within strings
showtabs=false,                 % show tabs within strings adding particular underscores
frame=single,                   % adds a frame around the code
tabsize=2,                      % sets default tabsize to 2 spaces
}

\renewcommand{\solutiontitle}{\noindent\textbf{Solution:}\par\noindent}

\newcommand{\mathsym}[1]{{}}
\newcommand{\unicode}{{}}
\newcommand{\tr}[1]{\tilde{#1}}
\newcommand{\LapTran}[1]{\mathscr{L}\left[#1\right]}
\newcommand{\InvLapTran}[1]{\mathscr{L}^{-1}\left[#1\right]}
\newcommand{\ParDer}[2]{\frac{\partial #1}{\partial #2}}
\newcommand{\Der}[2]{\frac{d\; #1}{d\;#2}}


\renewcommand{\DocTitle}{Syllabus}
\noprintanswers

%-------------------------------------------------------------------------------------
\begin{document}
%-------------------------------------------------------------------------------------
\begin{center}
\textbf{\DocTitle}
\end{center}
%-------------------------------------------------------------------------------------

\textbf{Instructor}: 
Lewis John Lloyd,  Engineering Research Building 841, Phone: 608/265-4056

\textbf{Times and Location}:
MWF 12:05 - 12:55,\ \  Material Science and Engineering 265

\textbf{Office Hours}:
MWF: 1:00-2:00 and by appointment.

\textbf{Required Textbook}:
\textit{Introductory Nuclear Reactor Dynamics}, Ott, K. O. and Neuhold, R. J., American Nuclear Society, 1985

\textbf{Prerequisite Courses}: 
NE405: Nuclear Reactor Theory

\textbf{Course Objectives}: 
This course is basic to nuclear reactor control and safety analysis.
It is designed to provide an understanding and use of the theory and methods of nuclear reactor dynamics, and to provide an exposure to analytic and numerical methods widely used in the analysis of reactor transients.
The areas of accident analysis and models, space-dependent effects, and methods of space-time analysis in reactor dynamics will be considered.

\textbf{Graded Events}
{
\begin{center}
\setlength{\extrarowheight}{2.0pt}
\begin{tabular}{lcc}
Event 			& Number 	& Total Weight \\ \hline
Problem Set 	& 6 		& .30\\
Quiz 			& 2 		& .40\\
Project 		& 1 		& .25\\
Participation 	& - 		& .05\\
\end{tabular}
\end{center}
}

\textbf{Grade Scale}
Baseline chart is given below; adjustments may be made based upon class performance.
\begin{center}
\begin{minipage}{0.4\linewidth}
\begin{tabular}{l@{\hspace{0.25\linewidth}}r}
A  & $          x \geq 93\%$\\
AB & $88\% \leq x < 93\%$\\
B  & $83\% \leq x < 88\%$\\
BC & $78\% \leq x < 83\%$\\
\end{tabular}
\end{minipage}
\hspace{0.05\linewidth}
\begin{minipage}{0.4\linewidth}
\begin{tabular}{l@{\hspace{0.25\linewidth}}r}
C  & $70\% \leq x < 78\%$\\
D  & $60\% \leq x < 70\%$\\
F  & $          x < 60\%$
\end{tabular}
\end{minipage}
\end{center}

\textbf{Late Policy}:
First, all problem sets must be turned in at the \underline{beginning of class} on the day they are due.
Second, all problem sets are \underline{mandatory}.
Late problem sets are divided into three categories and will be handled accordingly.

\begin{enumerate}
\item{
A coordinated late submission occurs when a student knows that they will miss the deadline for a graded event, and they contact me in advance. 
Note: notification immediately prior to the deadline, within 24 [hours], does not suffice.  
Penalties up to the amounts below \underline{may} be assessed for a coordinated late submissions.

{
\begin{center}
\setlength{\extrarowheight}{2.0pt}

\begin{tabular}{lcc}
Penalty & $\Delta T$ [hours] past deadline\\ \hline
12.5\%  & $0\;  \leq \Delta T < 12$ \\ 
25.0\%  & $12 \leq \Delta T < 24$\\
37.5\%  & $24 \leq \Delta T < 36$\\
50.0\%  & $36 \leq \Delta T < 48$\\
100.0\% & $48 \leq \Delta T < \infty$
\end{tabular}
\end{center}
}

} 

\item{An uncoordinated late submission occurs when a student misses a deadline but has failed to make prior arrangements with me. No credit will be given for an uncoordinated late submission; however, the assignment \underline{must still be submitted} for evaluation. Failure to submit all graded assignments will result in an incomplete until those assignments have been submitted.
}
\item{In cases where a student is confronted with unforeseen extenuating circumstances that impair their ability to meet a deadline, they should contact me as soon as possible to coordinate submission of the assignment.
}
\end{enumerate}

\textbf{Problem Sets}:
Throughout the semester, there will be six problem sets. Each problem set will be worth 5\% of the course grade.

\textbf{Quizzes}:
There will be two quizzes during the semester. Each quiz will be worth 20\% of the course grade. The quizzes will \underline{not} be open book; however, they will be based primarily (not entirely) on your conceptual understanding of the material covered. While technically not cumulative, the second quiz will require an understanding of material covered during the first half of the course.

\textbf{Project}:
As a whole, the project constitutes 25\% of the course grade. The project will be broken into three in-class presentations and a report. While each of the three presentations will be worth 5\%, the project report will be worth 10\%. See the project handout for additional details. 

\textbf{Final Exam}:
There will be \underline{no final exam} in this course.


\textbf{Academic Honesty}:
Students are allowed to work together on problem sets.
If a student receives help from sources other than their fellow classmates or myself, that assistance should be documented.
I expect students to adhere to the University standards regarding academic conduct and integrity:\\
\url{http://students.wisc.edu/saja/integrity.html}.\\
Should a student fail to meet these standards, disciplinary actions shall be pursued in accordance with University policies:\\
\url{http://students.wisc.edu/saja/misconduct/UWS14.html}.

\end{document}